
\subsection{RQ2 (All 19 properties, N=160): Do audit reports cluster into recognizable style families?}

\paragraph{Setup.} We re-ran the full RQ2 pipeline on all \textbf{19} canonical properties by
augmenting the 160-report matrix with text-derived signals from the additional spreadsheets you provided
(Report Structure, Methodology, Severity definitions, Knowledge). We parsed these fields using
regular expressions to detect the presence of each section (e.g., severity taxonomy, threat/trust model,
timeline, versioning, governance controls), and merged them with the original binary codings via the report link.

\paragraph{Prevalence (all 19).} \textbf{Detailed Findings} is present in \emph{every} report (160/160), and is therefore
\underline{not} used to define families. High-coverage properties include \emph{Executive Summary} (151/160, 94.4\%),
\emph{Recommendations} (149/160, 93.1\%), \emph{Scope} (144/160, 90.0\%), \emph{Methodology} (135/160, 84.4\%),
\emph{Status/Verification} (115/160, 71.9\%), and \emph{Tools \& Automation} (108/160, 67.5\%). Mid-to-low coverage properties
include \emph{Limitations/Disclaimers} (85/160, 53.1\%), \emph{Testing/Coverage} (84/160, 52.5\%), \emph{Documentation Quality} (79/160, 49.4\%),
\emph{System/Protocol Overview} (44/160, 27.5\%), \emph{Severity Model} (42/160, 26.2\%), \emph{Findings Summary (Table)} (42/160, 26.2\%),
\emph{Risk Taxonomy/Terminology} (27/160, 16.9\%), \emph{Threat/Trust Model} (14/160, 8.8\%), \emph{Privileged Roles/Centralisation} (14/160, 8.8\%),
\emph{Timeline} (5/160, 3.1\%), \emph{Upgradeability/Governance Controls} (4/160, 2.5\%), and \emph{Versioning/References} (1/160, 0.6\%).

\paragraph{Concept lattice.} The lattice enumerates 14 closed intents. The highest-support patterns stack actionability and packaging on
top of the backbone: \{Detailed Findings; Status/Verification\} (support 115), \{Detailed Findings; Recommendations; Status/Verification\} (43),
\{Detailed Findings; Recommendations; Status/Verification; Tools\} (24), and \{Detailed Findings; Executive Summary; Recommendations; Status/Verification; Tools\} (23).
This indicates that remediation \& verification often co-occur and are frequently accompanied by tool-based engineering and executive packaging.

\paragraph{Implications.} A pruned basis (422 implications after redundancy filtering) highlights that communication and method sections entail
technical substance, e.g., \emph{Executive Summary} $\Rightarrow$ \emph{Detailed Findings} (151), \emph{Scope} $\Rightarrow$ \emph{Detailed Findings; Executive Summary} (144),
\emph{Methodology} $\Rightarrow$ \emph{Detailed Findings; Executive Summary} (135). In other words, packaging and methodology rarely appear without detailed analysis.

\paragraph{Style families (all 19, Detailed Findings excluded).} We define five non-exclusive families aligned to the full property set:
(i) \textbf{Executive-Packaged} = Executive Summary \emph{and} (Severity Model \emph{or} Findings Table);
(ii) \textbf{Core-Engineering} = System Overview \emph{and} (Methodology \emph{or} Tools \emph{or} Testing);
(iii) \textbf{Remediation-First} = Recommendations \emph{or} Status/Verification;
(iv) \textbf{Governance-Focused} = Threat/Trust \emph{or} Privileged Roles \emph{or} Upgradeability/Governance;
(v) \textbf{Legal/Taxonomy-Heavy} = Limitations/Disclaimers \emph{and} (Risk Taxonomy/Terminology \emph{or} Severity Model).
Coverage across N=160: Remediation-First 95.0\%, Executive-Packaged 43.1\%, Core-Engineering 27.5\%,
Legal/Taxonomy-Heavy 22.5\%, Governance-Focused 18.1\%. Reports commonly layer multiple families, consistent with the lattice.

\paragraph{Soft adoption by provider.} We compute per-provider soft ratios (fraction of each provider's reports matching a family).
Adoption is broad for Remediation-First, while Executive-Packaged and Core-Engineering vary by provider specialization.
Governance-Focused and Legal/Taxonomy-Heavy are less common and concentrated among a subset of providers.

\paragraph{Artifacts.} We release CSVs for the 160-report, 19-property context, prevalence, lattice concepts, implications and pruned basis,
family definitions and coverage, and provider$\times$family soft ratios, plus a heatmap figure. See supplemental material in the main text.
